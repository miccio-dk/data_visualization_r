\chapter{Analysis}
The following sections aim to elaborate on the Problem formulation and create a more solid base to use as a guideline and template during the actual development.
This analysis does not aspire to be exhaustive, but rather satisfactory in the level of detail required to establish the project.


\section{Milestone plan}
Here is the project milestone plan as formulated during the initial phases of the project.
This version of the plan lists a series of achievements deemed necessary to accomplish the major functionalities.
It is arranged in a progressive way, with a time axis roughly flowing vertically.
Each task depends from its sub-tasks and --- more indirectly --- on the previous ones within its hierarchal level.

% Please refer to the \nameref{ch:appx} for the final version of the milestone plan based on the actual progress.

\renewcommand{\labelitemi}{\textbullet}
\renewcommand{\labelitemii}{\textbullet}
\renewcommand{\labelitemiii}{\textbullet}
\renewcommand{\labelitemiv}{\textbullet}
\begin{itemize}
	\item Extract requirements and use cases
	\item Evaluate and choose remote hosting service
	\item Familiarize with necessary tools and languages
	\begin{itemize}
		\item R and RStudio
		\item Java and Android Studio
		\begin{itemize}
			\item Acquire device location
			\item Read and append content to the remote platform
		\end{itemize}
	\end{itemize}
	\item Software development
	\begin{itemize}
		\item Configure remote service authorizations and keys
		\item Implement Android application
		\item Implement R script for visualizing data
	\end{itemize}
	\item Integration and validation tests
	\item Compiling necessary documentation
\end{itemize}


\section{Requirements}
The \emph{software requirements specification} is the formal description of the system to be developed, thoroughly pointing out what the product is --- and is not --- expected to do.
It also helps assessing the extent of the endeavor in terms of workload, costs, and other resources.
The requirements are commonly laid out in agreement with the client, and provide the basis upon which the product or project is evaluated.
Following the \emph{Unified Software Development Process} --- upon which this analysis is loosely based --- the requirements can be classified in \emph{functional} and \emph{non-functional}.

As far as this project is concerned, the supervisors have not set any particular requirements apart from those implied in the project name and problem formulation.
The list of requirements is therefore mostly based on former knowledge, common sense and best practices within the industry.


\subsection{Functional requirements}\label{subsec:req_func}
This section defines specific behaviors or functionalities, and is the basis for the tests.
Requirements introduced by the modal verb \emph{shall} indicate mandatory conditions, whilst those introduced by the modal verb \emph{may} indicate optional features.
% data have to be uploaded on Google Sheets
% visualization has to be done in R

\begin{table}[H]
\centerfloat
\begin{tabular}{@{} m{6em} >{\small}l @{}}
    \toprule
    ID      & \normalfont{Description} \\
    \midrule
    SRS.F.1    & blabla \\
    SRS.F.2    & blabla \\
    SRS.F.3    & blabla \\
    SRS.F.4    & blabla \\
    SRS.F.5    & blabla \\
    SRS.F.6    & blabla \\
    \bottomrule
\end{tabular}
\caption{Software requirements specification: functional requirements}\label{tab:srs_fun}
\end{table}


\subsection{Non functional requirements}\label{subsec:req_nf}
This section specifies desired characteristics and qualities of the system.

\begin{table}[H]
\centerfloat
\begin{tabular}{@{} m{6em} >{\small}l @{}}
    \toprule
    ID      & \normalfont{Description} \\
    \midrule
    SRS.NF.1    & blabla\\
    SRS.NF.2    & blabla \\
    SRS.NF.3    & blabla \\
    SRS.NF.4    & All source code shall follow a consistent format style \\
    SRS.NF.5    & All source code shall be properly documented \\
    SRS.NF.6    & All source code shall be trivially buildable and executable on other platforms \\
    SRS.NF.7    & The released documentation shall be written in English \\
    \bottomrule
\end{tabular}
\caption{Software requirements specification: non-functional requirements}\label{tab:srs_nfun}
\end{table}


\section{Tools}
This section will cover various software tools that have been employed during the course of the project.
These include development tool and software components or libraries used by the implemented code.


\subsection{Android Studio}
Android Studio is the official IDE (integrated development environment) for native Android development.
It is distributed freely by Google for Windows, Mac OS X and Linux platforms, and it is based on emph{JetBrains' IntelliJ IDEA}, a proprietary IDE for Java.
It replaced emph{Eclipse Android Development Tools} as Google's primary IDE for Android application development.

The IDE provides a series of Android-specific tools and features.
The most notable ones are:

\begin{itemize}
	\item Code editor with intelligent completion, linter, and refactoring system
	\item Integrated debugger and emulator based on virtual machines
	\item Graphical layout editor and wizards for code generation of UIs and other common software components
\end{itemize}

% add screenshot


\subsection{RStudio}
RStudio is a free and open-source IDE for R, a programming language for statistical computing.
Several editions of the software exists: a emph{Desktop} one, available on Windows, OS X, and Linux, and a emph{Server} and emph{ServerPro} one, with allows access through web browser from several terminals.

The software comprises a text editor with code completion and syntax highlighting, an interactive command interpreter with built-in debug, command history, and data viewer, as well as a package manager.


\subsection{R packages}



\subsection{Other tools}
%git, github, atom
