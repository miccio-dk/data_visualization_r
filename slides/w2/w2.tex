\documentclass{beamer}
\usepackage{beamerthemesplit} % new 
\begin{document}
\title{Data Visualisation for R language} 
\author{Eren Can Gungor} 
\date{\today} 

\frame{\titlepage} 

\frame{\frametitle{Table of contents}\tableofcontents} 


\section{ R in big data} 
\frame{\frametitle{Google BigQuery} 
Google BigQuery works as big data warehouse.  This system uses a familiar  SQL type platform to handle the data and process it as soon as possible
They have a pay-as-you-go system. That means money depends on how much data you will access and process.
}
\subsection{R environment in Google BigQuery }
\frame{\frametitle{ Linking R with Google BigQuery}
Using R will be easy in Google Query.  There are already pre-made packages such as "assertthat" and "bigquery". Both packages has been specifically made for implementing R in Google Query. Both Packages has been made by Hadley Wickham.
}
\subsection{Setting up R for Google BigQuery Operations}
\frame{\frametitle{ Setting up R for Google BigQuery Operations}
 Following steps has been taking to set up the code. Those steps are ;
 \begin{itemize}
 \item Introducing R packages for BigQuery
 \item Installing the R packages
 \item Loading the R packages
 \end{itemize}
}
\subsection{Setting up Google BigQuery Platform}
\frame{\frametitle{ Setting up Google  BigQuery Platform}
Google Query needs membership to sign up. It has similar concept as Github. Following steps needs to be taken for setup ;
\begin{itemize}
\item  Creating a Cloud Platform Console
\item Enabling Billing for your project
\item Enabling the BigQuery API-- ( Note that it will be automatically enabled for new projects)
\end{itemize}

For enquiring for a query, 
\begin{itemize}
\item Compose a query
\item Test with any  desired data that you will be interested in.
\end{itemize}

\section{Example Codes for Setting up and Testing Google BigQuery} 
\subsection{For linking R with Google BigQuery }
Following code has to be implement to linking R with Google BigQuery. Following steps will help us to using R shell to access the data from Google BigQuery
\begin{itemize}
\item library(bigrquery, lib=loc="/Users/erengungor")
\item install.packages("assertthat",lib="/Users/erengungor")
\item  install.packages("bigrquery",lib="/Users/erengungor")
\item "put your project ID here"
\item sql select title,contributor username,comment FROM [publicdata samples.wikipedia]
\item data (query exec("publicdata","samples",sql,billing = project)
\end {itemize}
}
\end{document}
